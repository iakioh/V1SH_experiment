%----------------------------------------------------------------------------------------
% Essay
%----------------------------------------------------------------------------------------
\newpage
\setcounter{page}{1} % Sets counter of page to 1

\section{Introduction} 
% Introduction & background: The Introduction should frame the scientific issues that motivate the study. It should briefly indicate the objectives of the study and provide enough background information to clarify why the study was undertaken and what hypotheses are tested. An overview of the key publications in the field is essential. 

%- Scientific issue: How objects recognized and localized? One strategy: localize border between textures. Neural mechanisms?
%- \cite{li_can_2000} model: neural circuits in V1 through iso-oriented inhibition and co-linear excitation, see Fig.\ \ref{fig:model}.I
%- \cite{popple_testing_2001} experiment: measured border localization inconsistent with bias predicitons by model
%-> limited in stimulus duration without visual backward masking -> confounding feedback from higher visual areas \cite{popple_testing_2001, liu_testing_2019, something_about_feedback_speed?}
%-> limited in central viewing -> confounding feedback from higher visual areas \cite{zhaoping_feedback_2017}
%- objectives of study: 
%1. replicate V1 model
%2. test predictions of model
%3. replicate and adapt experiment for borders in periphery, shown only 20ms with visual backwards masking
%4. collect preliminary data and compare with model

\paragraph{Background} How does the human brain recognize and localize objects? One strategy is to localize the borders between distinct visual textures. A neural circuit model of primary visual cortex (V1) by \citep{li_can_2000} suggests that texture borders can be localized through lateral neural interactions in V1, specifically iso-oriented inhibition and co-linear excitation (see Fig.\ \ref{fig:model}.I). However, \citep{popple_testing_2001} measured border localizations which are inconsistent with the biases predicted by the V1 model. So far, it remained unclear if the prolonged viewing duration of 150 ms, without visual backward masking, in \citep{popple_testing_2001}'s experiment allowed confounding feedback from visual areas beyond V1 \cite{liu_testing_2019}. And according to the Central-Peripheral-Dichotomy, central viewing of the stimuli in \citep{popple_testing_2001}'s experiment allows further confounding feedback \cite{zhaoping_feedback_2017}. 

\paragraph{Objectives} The objective of this lab rotation was to replicate the V1 model by \citep{li_can_2000}. I subsequently aimed to adapt \citep{popple_testing_2001}'s experiment to shorter stimuli durations viewed in the periphery and with visual backward masking. In a pilot experiment, I collected preliminary data on myself to test the experimental design and falsify predictions simulated with the replicated V1 model.  

% - Helmholtz illusion: if so, TODO short literature review on explanations and neural mechanisms of visual size estimation

\section{Methods}
% The Materials & Methods section should be brief but adequate to allow a qualified investigator to repeat the research. The equipment used should be described in detail. All methods of analysis and statistical testing must be identified and explained in detail. 

%1. V1 model replication:
%- implemented available at \url{https://github.com/iakioh/V1SH_model}
%- using Euler method for simulation time of $T = 12 \alpha$ where $\alpha$ denotes membrane time constant, with a typical step size of $\delta t = 0.01 \alpha$
%- model parameters as documented for original model \cite{li_neural_1998, zhaoping_understanding_2014}
%- exception: orientation-unspecific normalization for excitatory pyramidal cell, following notation with location $i$ and tuned to orientation $\theta$ as 
%\begin{equation}
%	I_o = 0.85 - 2.0 \left[ \frac{\sum_{j \in S_i} \sum_{\theta'} g_{x}(x_{j\theta'} ) }{C} \right]
%\end{equation}
%with $C = 16$ following the original implementation, not documentation where $C = \sum_{j \in S_i} 1 = 25$ using a Manhatten grid for sampling visual input
%- initial condition: resting state for no visual input (without top-down feedback)
%\begin{align}
%	y_0 &= I_c / \alpha = 1.0 \\
%	x_0 &= \left( I_o - g_y(y_0) \cdot \sum_{\theta'} \psi(\theta') \right) / \alpha = - 1.65
%\end{align}

\paragraph{Model replication} The V1 model (see Fig.\ \ref{fig:model}.I) was implemented using the documentation of the original model \cite{li_neural_1998, zhaoping_understanding_2014} and is available at \url{https://github.com/iakioh/V1SH_model}. For simulations I used the Euler method with a simulation time of $T = 12 \alpha$, where $\alpha$ denotes the membrane time constant, and a typical step size of $\Delta t = 0.01 \alpha$. I used the same model parameters as the original model, with one exception: The orientation-unspecific normalization for an excitatory pyramidal cell at location $i$ tuned to orientation $\theta$ (following the original notation) was computed as 
\begin{equation}
	I_o = 0.85 - 2.0 \left[ \frac{\sum_{j \in S_i} \sum_{\theta'} g_{x}(x_{j\theta'} ) }{C} \right]
\end{equation}
with $C = 16$ as in the original implementation of the V1 model, not the documentation where $C = \sum_{j \in S_i} 1 = 25$ when using a Manhatten grid for sampling visual input. Simulations started from the initial condition 
\begin{align}
	y_0 &= I_c / \alpha = 1.0 \\
	x_0 &= \left( I_o - g_y(y_0) \cdot \sum_{\theta'} \psi(\theta') \right) / \alpha = - 1.65
\end{align}
which corresponds to the neural resting state with neither visual input nor top-down feedback.

%2. predictions:
%- border saliency = highest saliency of grid column close, i.e.\ maximally 3 columns distance, to border. Saliency of grid columns = average saliency of all column locations. Saliency of a location = highest (temporally averaged) pyramidal response to inputs at this location. Z-scored over all locations $i$.
%- border localization bias = distance between most salient column close to border and actual border, i.e.\ midpoint between textures. Positive if bias towards orientation texture parallel to border, else negative. 
%- predict: 
%	- lower bias aomplitude and bias sign switch for decreasing number of rows
%	- lower saliency and thus lower sensitivity for decreasing number of rows
\paragraph{Model predictions} The saliency of a vertical border between two orientation textures, one oriented horizontally and the other vertically, was defined as the highest saliency of a grid column close, i.e.\ of maximally 3 columns horizontally distant, to the horizontal midpoint between two orientation textures. The saliency of a grid column was computed as the average saliency of all column locations. The saliency of a location is defined as the highest (temporally averaged) pyramidal response to visual input at this location \cite{zhaoping_understanding_2014}. The saliency of a border was normalized by z-scoring with the saliency distribution of all locations of the input texture.
The predicted localization bias for a vertical border was defined as the horizontal distance between the most salient column close to the border and the horizontal midpoint between the textures. By convention, the bias is positive if the most salient column is part of the vertically oriented texture, and negative if the most salient column is part of the horizontally oriented texture.

%3. experimental design:
%- closely follow \cite{popple_testing_2001}, implemented in PsychoPy \cite{...} at \url{https://github.com/iakioh/V1SH_experiment}
%- see Fig.\ \ref{fig:experiment}.I
%- pilot experiment: one subject, the experimentor himself, with corrected to normal visual acuity participated, therefore not naive to its purpose.
%- Stimuli generated by and viewed on a HP ProBook 450 G8 Notebook PC ($60$ Hz frame rate), from distance of $60$ cm 
%- Text prompt 
%	- to press any button to start next trial
%	- same location as fixation target appearing after button press
%	- framed by white ellipsoid outline of $10°$ width and $5°$ height, remained on screen throughout trial
%- Fixation: 
%	- "press any button to start next trial" prompt before at same position
%	- white circle with diameter of $0.1°$
%	- presented for $700$ ms
%- Stimulus: 
%	- Gabor patches: luminance $l \propto \exp(-(x^2 + y^2) / 2 \sigma^2) \cdot \sin(2 \pi f x)$ from mid-grey background ($x$: displacement from Gabor center along axis of sinusoidal modulation, $y$: along perpendicular axis)
%		- $100$\% Mitchell contrast
%		- $f = 3$ cpd
%		- $\sigma = 0.16°$
%		- * size motivated by \cite{popple_testing_2001} scaled up to account for larger receptive fields / lower resolution in the periphery by a factor of $(1 + \frac{e}{e_2}) \approx 4$ (rounded up) for border eccentricity $e = \sqrt{e_x^2 + e_y^2} \approx 11°$ and $e_2 = 3.3°$ \cite{zhaoping_feedback_2017}
%	- aranged in square grid texture with spacing of $6 \sigma$
%	- number of rows varied randomly between 1 or around 6 (upper texture) and 11 (lower texture) rows extending over the whole screen vertically
%	- around 32 columns extending over the whole screen horizontally
%	- uniform randomly jittered patch position by up to $\pm 0.6 \sigma$ in vertical and horizontal direction
%	- making up two orientation textures vertically separated by $1.72°$
%	- in center of each texture was vertical border defined by $90°$ orientation contrast
%	- upper and lower textures mirror images of one another, e.g. if upper texture contained vertical patches left of border and horizontal patches right of border, then lower texture contained horizontal patches left of border and vertical patches right of border
%	- orientation of Gabor patches varied randomly between $0°$ vs. $90°$ (vertical vs. horizontal upper right texture)
%	- horizontal offset of upper texture relative to lower texture shifted between trials
%	- border located in right or left upper peripheral visual field (to avoid any confounding effects of the blind spot), shifted by an eccentricity of approx.\ $\pm10°$ horizontally and $+5°$ vertically from fixation 
%	- fixation remained on screen on top of stimulus, as well as ellipsoid outline framing a stimulus exclusion zone to avoid saliently appearing stimuli within central vision
%- Mask: 
%	- Stimulus presented in this pilot experiment for $100 ms$
%	- followed by visual backward mask until subject response
%	- = grid of gabor patches of identical position as stimulus, but randomized orientations
%- Task: 
% 	- indicate on each trial whether upper border left or right of lower border through key press
%- Trials: 
%	- blocked in 15 minutes by border location in left vs right peripheral field, ... to ... trials per subject and ... trials in total
%	- following ... instruction and ... practice trials 
%	- horinzontal offsets selected via adaptive staircase procedure: QUEST+ \cite{watson_questplus_2017, python_questplus_package}
%		- scanned horizontal offsets in $[-3, 3]$ in steps of $0.25$ columns
%		- uniform prior
%		- cumulative normal likelihood
%			$$ \psi(d |\mu, \sigma, \delta_{left}, \delta_{right}) = \delta_{left} + (1 - \delta_{left} - \delta_{right}) \int_{-\infty}^{d} \mathcal{N}(x | \mu, \sigma) dx $$ \label{eq:CNF}
%			with $d$ the horizontal offset towards the vertically oriented upper texture, $\mu$ the mean, $\sigma$ the standard deviation or slope, and $\delta_{left}, \delta_{right}$ the left and right asymptote, also called lapse or guess rate
%		- posterior over parameters $\mu \in [-2.5, 2.5]$ in steps of $0.25$ columns, $\sigma \in [0.1, 5]$ in steps of $0.1$ columns, and $\delta \in [0, 0.45]$ in steps of $0.05$
%		- updated after each trial, horizontal offset selected to maximize expected information gain (minimize expected entropy of posterior after next trial)
%		- per block one staircase for each number of rows condition
\paragraph{Experimental stimuli} The experiment, see Fig.\ \ref{fig:experiment}.A, was implemented in PsychoPy \cite{peirce_psychopy2_2019} and is available at \url{https://github.com/iakioh/V1SH_experiment}. In the pilot experiment one subject, the experimentator himself, with visual acuity corrected to normal, participated. Stimuli were generated by and viewed on a HP ProBook 450 G8 Notebook PC ($60$ Hz frame rate), from an approximate distance of $60$ cm. Each trial started with a text prompting the subject to press any button to start the next trial. The text was shown at the same location as the fixation target appearing after a button press. The text was framed by a white ellipsoid outline of $10°$ width and $5°$ height, which was shown on screen throughout the whole remaining trial. The subsequent fixation target was a white filled circle with diameter of $0.1°$ and was presented for $700$ ms. 
The stimulus consisted of horizontally and vertically oriented Gabor patches with luminance $l \propto \exp(-(x^2 + y^2) / 2 \sigma^2) \cdot \sin(2 \pi f x)$ from mid-grey background ($x$: displacement from Gabor center along axis of sinusoidal modulation, $y$: along perpendicular axis), with $100$\% Mitchell contrast, a frequency of $f = 3$ cpd and a standard deviation of $\sigma = 0.16°$ \footnote{The chosen size of the Gabor patches is motivated by \cite{popple_testing_2001} scaled up to account for lower peripheral visual acuity by a factor of $(1 + \frac{e}{e_2}) \approx 4$ (rounded up) for $e_2 = 3.3°$ and eccentricity of the border $e = \sqrt{e_x^2 + e_y^2} \approx 11°$  \cite{zhaoping_feedback_2017}}. The Gabor patches made up "orientation textures", i.e.\ Gabor patches of equal orientation were arranged in square grids (with spacing of $6 \sigma$). In total, the stimulus consisted of four orientation textures, referred to by their relative location on the display: one upper left, one upper right, one lower left and one lower right texture. The upper and lower textures were vertically separated by $1.72°$. The patches in the left and right textures were of opposite orientation, thus separated by a vertical border defined by a $90°$ orientation contrast. The upper and lower textures were mirror images of one another, e.g.\ if the upper right texture contained vertical patches, so the upper left texture contained horizontal patches, then the lower left texture contained vertical patches and the lower right texture horizontal patches; and vica verca. Together, the textures extended over the whole screen horizontally, here around 32 columns. The number of rows varied randomly between 1 or extending over the whole screen vertically, here around 6 and 11 rows for the upper and lower textures respectively. The position of the center of each Gabor patch was uniformly randomized by up to $\pm 0.6 \sigma$ in both vertical and horizontal direction. The horizontal displacement of the upper textures relative to the lower textures was shifted between trials. The border was located in the right or left upper peripheral visual field of the subject fixating on the fixation target, with an eccentricity of approx.\ $e_y = +5°$ vertically (to avoid any confounding effects of the blind spot) and $e_x \pm10°$ horizontally from fixation. The fixation target remained on screen on top of the stimulus, together with the white ellipsoid outline framing a stimulus exclusion zone to avoid saliently appearing stimuli within central vision. The stimulus was presented for $100 ms$, followed by a visual backward mask. The mask was a grid of Gabor patches of identical position as the stimulus, but with randomized orientations. The mask was shown until the subject performed the task. 

\paragraph{Experimental task} The task was to indicate on each trial whether the border between the upper left and right texture is located "left" or "right" of the border between the lower left and right texture through pressing the "$\leftarrow$" or "$\rightarrow$" key respectively. Trials were blocked in sessions of maximally 15 minutes duration, following 12 instruction and maximally 30 practice trials per session. During one session, all trials showed stimuli with the border location only in the left or only in the right peripheral field. In total, 8 sessions with 2823 test trials were collected, with 1411 trials for stimulus with textures consisting of single rows, and 1412 trials for stimulus with textures consisting of multiple rows extending over the whole screen. 

\paragraph{Adaptive staircase} The horizontal displacement of the upper texture border was selected during the experiment with the adaptive staircase procedure QUEST+ \cite{watson_questplus_2017, python_questplus_package}. I used two QUEST+ staircases per session, one for all responses and stimuli consisting of multiple texture rows (condition $i = 1$) and one for all responses and stimuli consisting of single texture rows (condition $i = 2$). QUEST+ scanned horizontal displacements $d_i \in [-3, 3]$ columns, in steps of $0.25$ columns. For each staircase the response $r_i$ is defined as $1$ if the subject located the texture border towards the upper vertical texture, i.e.\ if the upper right texture is vertical and the subject responds "right" or if the upper left texture is vertical and the subject responds "left". The response is $r_i = 0$ if the subject pressed the key in the opposite direction towards the upper horizontal texture. Then, each staircase used a binomial likelihood $r_i \sim \text{Binom}(p(d_i))$ parametrized by the psychometric function
\begin{align} \label{eq:CNF}
	p(d_i |\mu_i, \sigma_i, \delta_{lower, i}, \delta_{upper, i}) &= \delta_{lower, i} + (1 - \delta_{lower, i} - \delta_{upper, i}) \cdot \psi(d_i | \mu_i, \sigma_i) \\
	\psi(d | \mu_i, \sigma_i) &= \int_{-\infty}^{d} \mathcal{N}(x | \mu_i, \sigma_i) dx,
\end{align}
where $\mu_i$ denotes the mean, $\sigma_i$ the standard deviation and $\delta_{lower, i}, 1 - \delta_{upper, i}$ the lower and upper asymptotes, which are non-zero if there are random errors or guesses. The posterior was computed assuming a uniform prior for parameters $\mu_i \in [-2.5, 2.5]$ in steps of $0.25$ columns, $\sigma_i \in [0.1, 5]$ in steps of $0.1$ columns, and symmetric lapse rates of $\delta_{lower, i} = \delta_{upper, i} \in [0, 0.45]$ in steps of $0.05$. After each trial, QUEST+ updates the posterior with the collected displacement-response pair and selects the horizontal displacement for the next trial which maximizes the expected posterior information gain. In the last two sessions, the staircases were initialized with all displacement-response pairs from previous sessions. 

% 4. data analysis: 
%- pooled all data of the same number of texture rows over left and right peripheral field
%- Maximum likelihood fit of responses in dependence of horizontal distance of texture borders with Cumulative Gaussian distribution $\mathcal{N}(\mu, \sigma)$
%- PSE -> bias $b = - \mu / 2$
%- bootstrapping for confidence intervals
%- one-sided permutation test to compare parameters fitted to multiple vs single texture row
\paragraph{Data analysis} Data from all trials with stimuli of equal number of texture rows, regardless of the border being shown in the left or right peripheral visual field, were pooled together. As shown in Fig.\ \ref{fig:data}, for each texture row condition $i \in \{1, 2\}$, psychometric parameters $\mu_i$, $\sigma_i$, $\delta_{lower, i}$ and $\delta_{upper, i}$ were fitted to the relationship between responses $r_i \in \{0, 1\}$ and horizontal displacement $d_i$ by maximizing the likelihood parametrized by eq.\ \ref{eq:CNF} 
\begin{equation}
	\prod_{n = 1}^{N_i} p(d_{i, n} |\mu_i, \sigma_i, \delta_{lower, i}, \delta_{upper, i})^{r_{i, n}} \cdot (1 - p(d_{i, n} |\mu_i, \sigma_i, \delta_{lower, i}, \delta_{upper, i}))^{1-r_{i, n}}.
\end{equation} 
assuming independent binomial-distributed responses for all trials $n \in [1, N_i]$. The point of subjective equivalence (PSE) is defined as the horizontal displacement $d_{PSE, i}$ for which the upper texture border is with equal chance perceived towards the upper vertical texture and in the opposite direction towards the upper horizontal vertical texture (after accounting for guesses or errors independent of the stimulus displacement), $\psi(d_{PSE, i} | \mu_i, \sigma_i) = 0.5$, which implies $d_{PSE, i} = \mu_i$. The border localization bias is therefore $b_i = - \frac{\mu_i}{2}$. 95\% confidence intervals for maximum likelihood estimates of the psychometric parameters were estimated by bootstrapping with $B = 10000$ samples. The estimated parameters for single vs.\ multiple texture rows were compared using a one-sided permutation test.

\section{Results}
% The Results of the study should be laid out in a series of declarative paragraphs. Only results essential to establish the main points of the work should be included. Often the reporting of the results can be clearer if broken down into subsections. All figures and tables must be cited in the text, and must be numbered in the order of their text citation. Figure legends should be self-explanatory, without referring to the text. They should identify the material that is being illustrated, what is shown, and its significance. Each table should be identified by number and should have a title. The Results section should not include long passages about the rationale of the experiments (which belong in the Introduction), or the methods used (which belong in the Material and Methods), nor should it include justification or discussion of the results (which belong in the Discussion section). 

%1. Model replication:
%- successful replication of V1 model verified by visualization of all model parameters, including connections, and comparison with expectation and original model \cite{li_neural_1998} as well as replication of 
%	- model responses to calibration inputs, see Fig.\ \ref{fig:model}.II,
%	- filling-in and avoiding leaking-out \cite{li_computational_2001, zhaoping_understanding_2014}
%	- temporal evolution of responses to texture border \cite{zhaoping_understanding_2014}, 
%	- the figure ground and medial axis effect \cite{zhaoping_v1_2003, zhaoping_understanding_2014}
%	- summation curve for grating disc inputs showing cross-orientation enhancement \cite{zhaoping_v1_2003, zhaoping_understanding_2014},
%	- dynamical properties of reduced two-point EI network \cite{li_computational_1998, zhaoping_understanding_2014}
\paragraph{Model replication} The successful replication of the V1 model was verified by visualizing all model parameters, including the neural connections (see Fig.\ \ref{fig:model}.I), and comparison with expectation and original model \cite{li_neural_1998}. Further, the following were replicated:
\begin{itemize}
	\item model responses to calibration inputs, see Fig.\ \ref{fig:model}.II
	\item filling-in and avoiding leaking-out \cite{li_computational_2001, zhaoping_understanding_2014}
	\item temporal evolution of responses to texture border \cite{zhaoping_understanding_2014}
	\item the figure ground and medial axis effect \cite{zhaoping_v1_2003, zhaoping_understanding_2014}
	\item the summation curve for grating disc inputs showing cross-orientation enhancement \cite{zhaoping_v1_2003, zhaoping_understanding_2014}
	\item dynamical properties of the reduced two-point excitatory-inhibitory network \cite{li_computational_1998, zhaoping_understanding_2014}
\end{itemize}

%2. Predictions:
%- see Fig.\ \ref{fig:experiment}.B, for medium to high input intensities
%- for higher number of rows border localization bias positive, i.e.\ towards texture parallel to border, confirming simulation by \cite{li_can_2000}, but negative for lower number of rows; confirming simplified argument by \cite{popple_testing_2001}
%- additionally, saliency of border higher for higher number of rows, because for lower rows co-linear excitation of texture vertically oriented to border dominates over iso-oriented inhibition between small number of rows. independent of displacement of the texture. 
\paragraph{Model predictions} As shown in Fig.\ \ref{fig:experiment}.B, for medium to high input intensities, the V1 model predicts a border localization bias towards the vertical texture for high number of texture rows, but a border localization bias in opposite direction towards the horizontal texture for a lower number of texture rows, confirming the simplified argument by \cite{popple_testing_2001}. Therefore, the model predicts $b_1 > 0$, $b_2 < 0$ and $b_1 > b_2$. Additionally, the saliency of the texture border is higher for a higher number of texture rows. Intuitively, for lower numbers of texture rows, the co-linear excitation between the horizontal texture elements dominates over the iso-oriented inhibition between vertical texture elements. If a border is less salient, a subject might increasingly guess randomly, i.e.\ independent of the horizontal displacement. Or a subject might rather rely more on more salient aspects such as co-linearly excited horizontal texture elements further away from the border to perform the task, i.e.\ potentially dependent on the horizontal displacement. Therefore, for textures with less rows, the model predicts either increased asymptotes, $\delta_{lower, 1} + \delta_{upper, 1} < \delta_{lower, 2} + \delta_{upper, 2}$, or a decreased sensitivity for border localization,i.e.\ $\sigma_1 < \sigma_2$.

%3. Data:
%- border localization bias towards vertical texture $b \approx 0.42 \pm 0.06$ significantly above zero for textures with multiple rows
%- low lapse rates $\delta_{left} = 0.00^{+0.04}{-0.00}, \delta_{right} = 0.00^{+10^{-6}}-{-0.00}$ and slope $\sigma = 0.8 \pm 0.1$ for textures with multiple rows
%- significantly higher lapse rates ($p_0 < 10^{-5}$) for single row textures 
%- for $\delta_{right} \neq \delta_{left}$, no significant border localization bias $b \approx 0.1^{+0.5}_{-0.2}$ for single texture row, and significantly lower than for mutliple texture rows ($p_0 \approx 3 \cdot 10^{-4}$)
%- the latter result depends highly on the fit. assuming $\delta_{right} = \delta_{left}$ leads to a significant bias $b \approx 0.4 \pm 0.2$ for single row texture, not significantly different from multiple texture rows ($p_0 \approx 0.43$). 
\paragraph{Preliminary data} As shown in Fig.\ \ref{fig:experiment}, for textures with multiple rows I find a border localization bias towards the vertical texture $b_1 \approx 0.42 \pm 0.06$ significantly above zero, which is in agreement with the V1 model prediction and the results by \cite{popple_testing_2001}. Further, low asymptotes $\delta_{left, 1} = 0.00^{+0.04}_{-0.00}, \delta_{right, 1} = 0.00^{+10^{-6}}_{-0.00}$ and slope $\sigma_1 = 0.8 \pm 0.1$ indicate high saliency of the border for multiple texture rows. I find significantly higher asymptotes ($p_0 < 10^{-5}$) for single row textures (and no significant difference for slopes), also in agreement with the V1 model prediction. Assuming $\delta_{lower} \neq \delta_{upper}$, I find for a single texture row a border localization bias $b_2 \approx 0.1^{+0.5}_{-0.2}$ not significantly larger than zero, and significantly lower than for multiple texture rows ($p_0 \approx 3 \cdot 10^{-4}$). Both results are in agreement with the model predictions, but I find a dependence on the fit. Assuming instead $\delta_{lower} = \delta_{upper}$ leads to a bias $b_2 \approx 0.4 \pm 0.2$ significantly larger than zero for single texture rows, not significantly different from the bias $b_1$ for multiple texture rows ($p_0 \approx 0.43$). A preliminary comparison of the goodness of fit considering also the different number of fit parameters remains inconclusive, since for single texture rows the Aikake information criterion (AIC) gives a lower value for the $\delta_{lower} \neq \delta_{upper}$ model, but the Bayesian information criterion (BIC) the opposite. 

\section{Discussion} 
% Discussion & conclussions: The Discussion should begin with a statement of the important findings of the study. Subsequent sections can address technical issues, analysis of the results, and the implications of the work. Again, it is often helpful to break down the Discussion into subsections that focus on particular topics. It is proper to include a section that summarizes and expands upon conclusions that may be drawn from the work (raises open questions, proposes future studies). 

%- important findings:
%	- V1 model successfully replicated, with minor improvements to documentation
%	- confirmed switch in localization bias and predicted decreased saliency of texture border
%	- designed and implemented improved experiment to test this prediction
%	- tested experiment in and analyzed data of a pilot study, finding 
%		- significant localization border bias towards vertically oriented texture for textures with mulitple rows, in agreement with the predictions by V1 model and with \cite{popple_testing_2001}
%		- significantly higher lapse rate for textures of single row, in agreement with lower saliency predicted by V1 model

\paragraph{Summary} In this study, the V1 model by \citep{li_can_2000} was successfully replicated, with suggestions for minor improvements of the documentation. For orientation textures with vertical borders, I confirmed that for decreasing number of texture rows, the V1 model predicts a switch of a border localization bias from towards the vertical texture to the horizontal texture. I adapted and implemented an experiment to test this prediction with less confounding top-down feedback from visual areas beyond V1. In a pilot study, I find a significant localization border bias for textures with multiple rows, in agreement with the model predictions as well as the experimental results by \citep{popple_testing_2001}. I further find a significantly higher lapse rate for single texture rows, in agreement with the lower saliency predicted by the V1 model. 

\paragraph{Limitations} The V1 model predictions were made using a heuristic read-out of the saliency map for the texture border localization, and only for input with a single texture border, not based on an optimal observer solving the actual experimental task by sampling locations from the saliency distribution predicted by the V1 model for visual input consisting of two texture borders. Data was only collected for a single, non-naive subject. Further, while low lapse rates and precision of the bias measurement indicate that the experimental task, at least for multiple texture rows, was easy enough for one non-naive subject, preliminary data collected in a single session from a naive subject showed responses not higher than chance. This suggests that the task might be too difficult without training. 
%-> train or make easier, e.g. only shifting a single texture. possible confounding factor: non-naive subjects might use the fact that only upper texture border moves and approximate location of lower texture border -> also shift lower texture border
% - leaking-out in replicated and original model in dependence of numerical accuracy

% Further work / implications:
%	- further improvements of experimental design and data analysis to rebustly estimate bias for single row texture, more subjects, lower viewing durations of up to 20ms
%	- biases and sensitivity might falsify V1 model explanation -> support alternative neural mechanisms, e.g.\ in higher visual areas % TODO: maybe add additional sources since then!
%	- else, assuming visual size estimation consistent with border localization, model predicts visual size illusion inverse to well-known Helmholtz illusion \cite{helmholtz_handbuch_1867}
%	- future comparative study of both illusions for new experimental and theoretical avenue to understand contributions of V1 and higher visual areas to visual texture segmentation and size perception
\paragraph{Outlook} Further improvements of the experimental design and data analysis are necessary to robustly estimate the bias for single texture rows. More data collections with multiple subjects are needed. A further decreased viewing durations of up to 20 ms beyond the scope of this study might further decrease confounding feedback. The biases measured in future work might falsify the V1 model explanation and support alternative neural mechanisms of visual texture segmentation. Else, assuming a visual size estimation consistent withe border localization, the V1 model might predict a new visual size illusion inverse to the well-known Helmholtz illusion (1867). Future comparative study of both illusions might open up a new experimental and theoretical avenue to understand the contributions of V1and higher visual areas to not only visual texture segmentation but also size perception.

\section{Acknowledgements}
% These may include thanks to technicians or colleagues who have helped with the work or provided materials. 
%- supervision and funding by Li Zhaoping
%- support by fellow lab members Vladislav Aksiotis, Fani Lohrmann and Maria Pavlovic. 
I very much appreciated the support by from my supervisor Li Zhaoping, including the important scientific lessons and the exciting scientific input. I also thank fellow members of the Zhaoping Lab, especially Vladislav Aksiotis, Fani Lohrmann and Maria Pavlovic, for help with my many questions on psychophysical experiments and patient feedback on my work. Lastly, I thank all non-naive subjects for their patience with my experimental design. 

\begin{figure}[htbp!] 
	\centering
	\includegraphics[width=\linewidth]{figures/Figure_1.pdf}
	\caption{Model replication. I: Sketch of neural circuit model of primary visual cortex (V1). Top: The model consists of two cell types, excitatory pyramidal cells, which are the in- and output of the model, and inhibitory interneurons. Bottom: Orientation and location of bars depict tuning and receptive field of edge / bar detector neurons in V1.  The excitatory co-linear and inhibitory iso-oriented interactions mediate contextual influences underlying saliency computations in V1. II: Inputs and replicated outputs for model calibration. A-D show contextual effects dominated by iso-orientation supression. E-H show contextual effects dominated by co-linear facilitation.}
	\label{fig:model}
\end{figure}

\begin{figure}[htbp!] 
	\centering
	\includegraphics[width=\linewidth]{figures/Figure_2.pdf}
	\caption{Experimental design and predicitions. A: One example trial of the proposed experiment. After subjects fixate for 700 ms, the stimulus with two texture borders of varying horizontal displacement and number of rows is shown in the periphery for 100 ms. Visual backward masking afterwards reduces confounding extrastriatal feedback. Subjects are forced to decide between two alternative choices: is the upper texture border left or right of the lower texture border? B: Predicted saliency and localization of borders given orientation textures of varying height. The absolute value and sign of the x-axis show border saliency (z-scored, here found to be positive for all predictions) and the direction of the localization bias as indicated by the sketched texture border above the y-axis, respectively. The y-axis shows the number of rows of the input texture, with examplatory predicted saliency maps for the case of 3 and 100 rows. For medium to high input intensity and increasing number of rows, the border saliency increases after a sign switch of the border localization bias. Due to wrap-around boundary conditions and a model input size of 100 rows, 100 rows indicates the case of texture ends beyond the visual input field.}
	\label{fig:experiment}
\end{figure}

\begin{figure}[htbp!] 
	\centering
	\includegraphics[width=\linewidth]{figures/Figure_3_2.pdf}
	\caption{Preliminary experimental results. A-B: Psychometric fit for stimuli textures consisting of multiple rows (A) or a single row (B). The x-axes show the horizontal displacement of the upper textures varied during the experiment, in units of the texture column width. Positive displacements are towards the upper vertical texture, e.g.\ if the upper right texture is vertical, positive displacements are towards the right. The y-axes show the percentage of trials for which the pilot subject localizes the border towards the upper vertical texture, e.g.\ if the upper right texture is vertical, the percentage of trials pressing the key "right". Blue dots visualize the data, with diameter indicating the number of trials and error bars the Wilson score interval. Black line visualizes the maximum likelihood fit (MLE) of eq.\ \ref{eq:CNF}, grey outline indicates bootstrapped 95\% CI. Parameters of the fitted psychometric likelihood are indicated in red: mean $\mu$, standard deviation $\sigma$ and lower / upper asymptotes $\delta_{lower}, \delta_{upper}$. C-E: fitted psychometric parameters of A (stimulus condition $i = 1$) and B (stimulus condition $i = 2$) in comparison. The red dots indicate the maximum likelihood estimate (MLE). Boxplots visualize the bootstrap distribution. P-values $p_0$ for respective null hypotheses $H_0$. C: bias $b_i = -\mu_i / 2$, D: standard deviation $\sigma_i$, inversely proportional to slope. E: average asymptotes or lapse rates  $(\delta_{lower} + \delta_{upper}) / 2$. Important note of caution: the significance and results of B-D highly depend on the fit, especially the assumption $\delta_{lower} \neq \delta_{upper}$.}
	\label{fig:data}
\end{figure}
