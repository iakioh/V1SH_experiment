%---------------------------------------------------------------------------------
% Abstract
%---------------------------------------------------------------------------------

% The Abstract should be concise, consisting of 200 words or less. It should briefly frame the biological problem that the paper addresses, indicate in brief the method of approach, the species of animal used, and provide a concise summary of the major results and conclusions (no subheadings, no references to the literature). 

\setcounter{page}{0} % Sets counter of page to 1

\begin{abstract}
    How does the human brain recognize and localize objects? One strategy is to localize the borders between distinct visual textures. Li's (2000) neural circuit model of primary visual cortex (V1) proposes that texture borders are localized through lateral neural interactions in V1, specifically iso-oriented inhibition and co-linear excitation. However, Popple (2003) measured border localizations which are inconsistent with the biases predicted by the V1 model. So far, it remained unclear if the limitations of Popple's psychophysical experiment, specifically central viewing and prolonged stimulus duration of 150 ms, allowed confounding feedback from visual areas beyond V1. In this experiment we test if the biases in localizing borders between orientation textures contradict the V1 model even in the case of stimuli presented for durations as short as 20 ms in peripheral vision. We replicate and extend Popple's experiment, and compare our results with predictions simulated by a replicated version of the V1 model. The biases measured in this experiment might falsify the V1 model explanation and support alternative neural mechanisms of visual texture segmentation. Else, the V1 model and our results predict a new visual size illusion inverse to the well-known Helmholtz illusion (1867). Future comparative study of both illusions might open up a new experimental and theoretical avenue to understand the contributions of V1 and higher visual areas to not only visual texture segmentation but also size perception.
\end{abstract}